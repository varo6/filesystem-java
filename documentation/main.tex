\documentclass[a4paper, 12pt]{report}

%%%%%%%%%%%%
% Packages %
%%%%%%%%%%%%

\usepackage[spanish]{babel}
\usepackage{packages/sleek}
\usepackage{packages/sleek-title}
\usepackage{packages/sleek-theorems}
\usepackage{packages/sleek-listings}
\usepackage{dirtree}
\usepackage{tikz}
\usepackage{pgfplots}
\usepackage{pgfplotstable}




%%%%%%%%%%%%%%
% Title-page %
%%%%%%%%%%%%%%

\logo{./resources/pdf/logo.pdf}
\institute{Universidad Politécnica de Cartagena}
\faculty{Ingeniería Telemática}
%\department{Department of Anything but Psychology}
\title{Conexión Cliente-Servidor mediante sockets en Java}
\subtitle{Trabajo de prácticas de sistemas distribuidos }
\author{\textit{Autores}\\\textsc{Álvaro Herández Riquelme}\\ y \textsc{André Yermak Naumenko}}
%\supervisor{Linus \textsc{Torvalds}}
%\context{A long time ago in a galaxy far, far away...}
\date{\today}

%%%%%%%%%%%%%%%%
% Bibliography %
%%%%%%%%%%%%%%%%

\addbibresource{./resources/bib/references.bib}

%%%%%%%%%%
% Macros %
%%%%%%%%%%

\def\tbs{\textbackslash}

%%%%%%%%%%%%
% Document %
%%%%%%%%%%%%

\begin{document}
    \maketitle
    \romantableofcontents

\newpage

    \chapter{Introducción}

    En este documento se presenta el trabajo realizado en la asignatura de sistemas distribuidos, en el cual se ha
    implementado un sistema de transferencia de archivos entre un cliente y un servidor mediante sockets en Java.
    Se ha elegido el uso de sockets para la comunicación entre el cliente y el servidor, ya que lo consideramos una
    forma más sencilla y eficiente de implementar lo que se pide en estre trabajo.

    \section{Estructura del proyecto}

    La estructura del proyecto se basa mayoritariamente alrededor de \textbf{la máquina de estados} que hemos
    diseñado para éste, siendo de gran importancia el contexto que tenga el programa en todo momento, ya sea
    cliente, servidor, o system, el contexto que tiene la máquina antes de poder pasar al contexto de cliente o
    servidor.

    \begin{figure}[htb]
        \centering
        \begin{tikzpicture}[]
            \node[state] (s1) {Sistema};
            \node[state, below right of=s1] (s2) {Cliente};
            \node[state, below left of=s1] (s3) {Servidor};

            \draw
            (s1) edge[bend left]    (s2)
            (s1) edge[bend right]   (s3)
            (s2) edge[bend left]    (s1)
            (s3) edge[bend right]   (s1);
        \end{tikzpicture}
        \caption{Máquina de estados diseñada para el proyecto.}
        \label{fig:diagrama}
    \end{figure}

    Teniendo en cuenta la maquina de estados (funciona tal tal....), la estructura de archivos finalmente quedará
    organizada de la siguiente manera:

    \dirtree{%
        .1 filetransfer.
        .2 FileSystem.java.
        .2 Main.java.
        .2 SystemContextHandler.java.
        .2 client.
        .3 ClientContextHandler.java.
        .3 ClientMain.java.
        .3 ClientUtils.java.
        .3 SimpleClient.java.
        .2 common.
        .3 CloseMessage.java.
        .3 CommandMessage.java.
        .3 ConsoleGUI.java.
        .3 Const.java.
        .3 Context.java.
        .3 ContextCommandHandler.java.
        .3 ContextManager.java.
        .3 ContextObserver.java.
        .3 Header.java.
        .3 TextMessage.java.
        .3 UserMessage.java.
        .3 Utils.java.
        .2 META-INF.
        .3 MANIFEST.MF.
        .2 server.
        .3 ConcurrentServer.java.
        .3 ServerCommandProcess.java.
        .3 ServerContextHandler.java.
        .3 ServerMain.java.
    }


    \chapter{filetransfer}
    \section{FileSystem.java}
    \section{Main.java}
    \section{SystemContextHandler.java}

    \chapter{common}
    \section{CloseMessage.java}
    \section{CommandMessage.java}
    \section{ConsoleGUI.java}
    \section{Const.java}
    \section{Context.java}
    \section{ContextCommandHandler.java}
    \section{ContextManager.java}
    \section{ContextObserver.java}
    \section{Header.java}
    \section{TextMessage.java}
    \section{UserMessage.java}
    \section{Utils.java}

    \chapter{client}
    \section{ClientContextHandler.java}
    \section{ClientMain.java}
    \section{ClientUtils.java}
    \section{SimpleClient.java}

    \chapter{server}
    \section{ConcurrentServer.java}
    \section{ServerCommandProcess.java}
    El ServerCommandProcess, es el encargado de procesar los objetos \textbf{CommandMessage} que recibe el
    servidor, y actuar en consecuencia. Cada comando requiere una función propia, y al optar
    por hacerlo con un enfoque distinto al de clase, donde cada comando podría ser una clase distinta y que cada
    función se ejecute según el objeto, se ha hecho un \textbf{hashmap} que contiene las funciones que se ejecutarán
    según el comando que se reciba.

    Todas las funciones que se ejecutan en el hashmap, reciben un objeto de tipo cm,
    que será el comando enviado por el cliente, ya habiendo pasado las validaciones tanto de tipado (el casting),
    como de argumentos (En la parte del cliente). Cada función del hashmap también recibe un objeto de tipo
    \texbt{Path}, del paquete de java.nio, que será el directorio en el que se ejecutará el comando, el cual se
    procesa al inicio de la clase. Este path es una combinación del basepath, donde se encontrará storage, y el
    clientpath, que será el directorio en el el servidor entiende que se encuentra el cliente. La función
    \textbf{directoryOpen} es la que actualiza el clientpath, y se ejecuta cuando el comando del cliente es
    \textbf{cd}. Se tiene en cuenta que el cliente no pueda salir del directorio base.

    \section{ServerContextHandler.java}
    \section{ServerMain.java}

\end{document}
